\documentclass[oribibl]{styles/llncs}

\usepackage[utf8]{inputenc}
\usepackage[ngerman]{babel}


\title{SLAM Simultaneous Localization and Mapping}
\author{}
\institute{FH Aachen University of Applied Sciences}

\begin{document}
\maketitle

%\citet{Durrant-Whyte, H. and Bailey, Tim}
SLAM steht für "`Simultaneous Localization and Mapping"' was man als "`Simultane Lokalisierung und Kartografierung"' übersetzen kann. \\
SLAM gilt als der "`Heilige Gral"' der Robotik \cite{1638022}, da die Lösung diese Problems einem Roboter erlaubt sich in einer fremden Umgebung zu orientieren. \\
Dazu erstellt der Roboter, mit Hilfe sogenannter Landmarken, eine Karte von seiner Umgebung und bestimmt seine Position in dieser Karte.
Da dies geschieht simultan, daher auch die Bezeichnung SLAM.


\section{Legende}
\begin{itemize}
	\item $\mathbf{x}_k$: ist ein Status-Vektor der die Position und Ausrichtung zum Zeitpunkt k speichert.
	\item $\mathbf{u}_k$: ist ein Kontroll-Vektor der beschreibt, wie der Roboter vom Status $\mathbf{x}_{k-1}$ zum Status $\mathbf{x}_k$
	\item $\mathbf{m}_i$: ist ein Vektor der die Position der i-ten Landmarke beschreibt.
	\item $\mathbf{z}_{ik}$: ist ein Vektor der in Richtung der i-ten Landmarke zum Zeitpunkt k zeigt.	\\ \\
Zusammenfassend ergeben diese Vektoren die Matrizen:
	\item $\mathbf{X}_{0:k} = \{\mathbf{x}_0, \mathbf{x}_1, ..., \mathbf{x}_k\}$: ist die Historie aller Position des Roboters.
	\item $\mathbf{U}_{0:k} = \{\mathbf{u}_0, \mathbf{u}_1, ..., \mathbf{u}_k\}$: ist die Historie der Kontroll-Vektoren.
	\item $\mathbf{m} = \{\mathbf{m}_0, \mathbf{m}_1, ..., \mathbf{m}_n\}$: alle entdeckten Landmarken.
	\item $\mathbf{Z}_{0:k} = \{\mathbf{z}_0, \mathbf{z}_1, ..., \mathbf{z}_k\}$: ist die Historie aller z Vektoren.
\end{itemize}
\cite{1638022}



\bibliographystyle{styles/splncs03}
\bibliography{citations}
\end{document}